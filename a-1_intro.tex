\section*{A-1. はじめに}
混雑度は日ごろの生活環境の重要な情報の一つである.
観光地や商業施設,交通機関において,混雑度は利用者の行動に影響を与えると同時に,
サービスの提供者側にとって,サービス改善を図るための重要な指標の一つとなる.
さらにCOVID-19の流行により,一部空間の過剰な混雑を避ける重要性が高まった.
いずれの観点においても,人々の生活空間において「混雑度を可視化」することの重要性は高い.
一方で混雑度を人間が定量的に評価することは難しい.
これに対して,様々な人数推定手法の研究が進められてきた.
例えば,カメラを用いた方法などが提案され\cite{camera_crowd},実用化されている例も多く存在する.
しかし,この手法ではカメラに映る人々の映像が解析に使われることから,
プライバシー侵害の危険性が示唆されている\cite{senkou_privacy}.
他に,Wi-FiやBluetooth Low Energy(BLE)などの電波を使う方法が提案されている.
BLEを用いた人数推定では,スマートフォンなどから発せられるBLEの電波をスキャンすることで,
その受信状況から特定空間の人数を推定する.
路線バスや電車などの交通機関で検証された例では,一定の有効性が示されている.
さらに,公共施設や飲食店での研究例も報告されている\cite{senkou}.
最大で50席ほどの空間で人数が推定され,一定の有効性が確認されている.
しかし,これらの研究例は,人数推定の対象が比較的小規模な空間に限られている.
広いショッピングモールの空間やイベント会場などでの,大規模な空間での応用場面が考慮された
研究例は少なく,人数推定の精度や有効性がどの程度保証されるかは不明である.
本プロジェクトでは,最大400席程ある学生食堂において,
BLEを用いた人数推定手法の有効性を検証した.
これにより,大規模な空間での人数推定にBLEを用いた手法が一定の有効性がある事が示された.
さらに人数推定に有用な特徴量や推定モデルについて議論し,より高精度な人数推定手法の提案を行った.
本プロジェクトの貢献は以下の通りである.
\begin{itemize}
  \item BLEを用いた人数推定手法の有効性を,最大400席の大規模な空間で検証した.
  \item RSSI値を用いた人数推定手法が,開けた空間のBLEスキャンデータにおいて,RSSI値は特徴量として一定の有効性があることを示した.
\end{itemize}