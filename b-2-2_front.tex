\subsection*{2. フロントエンドの工夫点}

続いて,フロントエンドの工夫点について述べる.

\subsubsection*{(1)Reactの採用}
フロントエンド開発には,JavaScriptライブラリであるReactを採用した.Reactは,以下の点で本システムに適していると判断した.

\begin{itemize}
	\item コンポーネントベースのUI設計
	
	UIをコンポーネント単位で分割して開発することで,コードの再利用性が高まり,メンテナンス性も向上した.また,機能ごとに責務を分離することで,チーム開発における分業が容易になった.
	
	\item 宣言的なパラダイム
	
	手続き型のコードとは異なるシンプルな記述により,コードの可読性が向上した.状態の変化に応じてUIがどのように変化するべきかを宣言的に記述できるため,複雑なDOMの操作を直接行う必要がなくなった.
	
	\item 効率的なレンダリング
	
	仮想DOM(Virtual DOM)の採用により,ブラウザで表示するUIを効率的に(最低限の描画で)レンダリングすることが可能になった.これにより,アプリケーションのパフォーマンスが向上し,ユーザーエクスペリエンスが改善された.
\end{itemize}

\subsubsection*{(2)レスポンシブデザインの実装}
本システムでは,様々なデバイスからのアクセスを想定し,ブラウザで表示するUIをレスポンシブデザインで設計した.これにより,デスクトップPCからスマートフォンまで,異なる画面サイズに対応したユーザーインターフェースを提供することが可能になった.

図\ref{fig:responsive_design}に示すように,デバイスのサイズに応じてレイアウトが自動的に調整され,どのデバイスからもストレスなく操作できる環境を実現した.

% ここに図を入れる場合
% \begin{figure}[tb]
%	\centering
%	\includegraphics[width=0.8\linewidth]{responsive_design.png}
%	\caption{レスポンシブデザインの例}
%	\label{fig:responsive_design}
% \end{figure}

\subsubsection*{(3)動的なUI表現の実装}
本システムでは,ユーザーエクスペリエンスを向上させるため,以下のような動的なUI表現を実装した.

\begin{itemize}
	\item 混雑度表示のアニメーション化
	
	システムが推定した混雑度を,単なる数値や静的なグラフィックではなく,動的なアニメーションで表現することで,より直感的に混雑状況を把握できるようにした.これにより,ユーザーは一目で現在の混雑状況を理解することが可能になった.
	
	\item 更新機能の簡素化
	
	更新ボタンを設計することにより,予測時刻の更新を簡素化した.ユーザーはボタン一つで最新の混雑情報を取得でき,操作の煩雑さを排除することで,ユーザーエクスペリエンスの向上を図った.
	
\end{itemize}

表\ref{tbl:UI_components}に,主要なUIコンポーネントとその役割について示す.

\begin{table}[tb]
	\centering
	\caption{主要なUIコンポーネント}
	\label{tbl:UI_components}
	\small
	\doublerulesep=0.3pt
	\begin{tabular}{l|p{9cm}} \hline\hline\hline
		コンポーネント名 & 役割 \\ \hline
		CongestionDisplay & 混雑度をアニメーションと数値で表示するコンポーネント \\ \hline
		UpdateButton & 最新の予測データを取得するためのボタンコンポーネント \\ \hline
		TimeStampDisplay & 予測時刻を表示するコンポーネント \\ \hline
		ResponsiveContainer & 画面サイズに応じてレイアウトを調整するコンテナコンポーネント \\ \hline\hline\hline
	\end{tabular}
\end{table}