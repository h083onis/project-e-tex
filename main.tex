\documentclass[twocolumn]{jarticle}

\usepackage[dvipdfmx]{graphicx}
\usepackage{amsmath}
\usepackage{amssymb}
\usepackage{amsfonts}
\usepackage{jsaiac}
\usepackage{array}
\usepackage{multirow}
\usepackage{booktabs}
\usepackage{ascmac}

\usepackage{listings}
\lstdefinestyle{mystyle}{
    basicstyle=\ttfamily\footnotesize,
    stepnumber=1,
    frame=single,
    breaklines=true,
    captionpos=b
}

\def\thline{\noalign{\hrule height 1.3pt}}
\def\tvline{\vrule width 1.3pt}

\title{
\jtitle{
\\BLEを用いた食堂の混雑度センシングに関する研究報告
\\~誰もが不便なく利用できる学食を目指して~
}
}

\jaddress{m807040z@mails.cc.ehime-u.ac.jp}

\author{
\jname{平本宗大 上村航平 大西真輝 佐野一樹 高屋友輔 平木晶 三好涼太 山下智也 吉原駿平}
}

\affiliate{
\jname{理工学研究科 数理情報プログラム}
}

\begin{abstract}
本プロジェクトは,2つの側面を持つプロジェクトである.
1つ目は,学術的な観点からBluetooth Low Energy (BLE)を用いた混雑度推定についての研究を行うことである.
2つ目は,学生食堂において,BLEを用いた混雑度推定システムを実装し,ICT技術を用いて,実際の課題解決を目指すことである.
この2つの側面を一気通貫して1つのプロジェクトとして進める.
我々は,このプロジェクトを通して,学術的な成果を即座に実社会に実装できることを証明した.
このような取り組みを積極的に行うことは,より高度化するICT社会において,
研究から生まれた成果をスピード感をもって社会に還元することに必要不可欠であると考える.
この報告書において,プロジェクトの学術的な側面についてはA-1からA-8,その実装についてはB-1からB-3に分けて報告する.

\end{abstract}

\def\Style{``jsaiac.sty''}
\def\BibTeX{{\rm B\kern-.05em{\sc i\kern-.025em b}\kern-.08em%
 T\kern-.1667em\lower.7ex\hbox{E}\kern-.125emX}}
\def\JBibTeX{\leavevmode\lower .6ex\hbox{J}\kern-0.15em\BibTeX}
\def\LaTeXe{\LaTeX\kern.15em2$_{\textstyle\varepsilon}$}

\begin{document}
\maketitle
  \chapter*{A-1. はじめに}
\section*{A-2. 関連研究}
\section*{A-3. 提案手法}
\section*{A-4. 実験}
本章では,結果と考察について述べる.5.1節で結果について述べ,5.2節で考察について述べる.

\section*{A-6. 結果}

\subsection*{6.1 モデル性能に関する結果}
本項では,5種類の回帰モデル(SVR, RFR, XGBR, LGBM, CatBoost)に対する性能評価結果を示す.まず,各モデルにおけるデフォルトのパラメータでの実験結果を表\ref{tab:default_results}に示す.次に,Optuna を用いて調整した各モデルの最適なハイパーパラメータを表\ref{tab:best_params}に示す.最後に,パラメータ調整後の各モデルの性能を表\ref{tab:tuned_results}に示す.
\begin{table}[htbp]
    \centering
    \doublerulesep=0.3pt
    \caption{デフォルトパラメータでのモデル評価結果}
    \label{tab:default_results}
    \begin{tabular}{l|p{0.8cm}p{0.8cm}p{0.8cm}|p{0.8cm}p{0.8cm}p{0.8cm}}
        \hline\hline\hline
        モデル & \multicolumn{3}{c|}{検証データ} & \multicolumn{3}{c}{テストデータ} \\
               & MSE & MAE & $R^2$ & MSE & MAE & $R^2$ \\
        \hline
        SVR       & 4042.62 & 41.93 & 0.584 & 4098.13 & 43.35 & 0.585 \\
        RFR       & 525.61  & 15.79 & 0.946 & 550.68  & 15.50 & 0.944 \\
        XGBR      & 547.63  & 15.01 & 0.944 & 618.45  & 15.21 & 0.937 \\
        LGBM      & 456.59  & 14.82 & 0.953 & 487.50  & 14.25 & 0.951 \\
        CatBoost  & 361.71  & 13.30 & 0.963 & 469.74  & 13.67 & 0.952 \\
        \hline\hline\hline
    \end{tabular}
\end{table}

\begin{table}[htbp]
    \centering
    \doublerulesep=0.3pt
    \caption{各モデルの最適パラメータ}
    \label{tab:best_params}
    \begin{tabular}{l|c}
        \hline\hline\hline
        モデル & 最適パラメータ \\
        \hline
        SVR & $C=0.093$ \\ 
            & $\epsilon=0.090$\\
            & $\text{kernel}=\text{linear}$ \\
        \hline
        RFR & $\text{max\_depth}=10$ \\
                     & $\text{min\_samples\_split}=2$ \\
                     & $\text{n\_estimators}=4000$ \\
        \hline
        XGBR & $\text{learning\_rate}=0.082$ \\
                & $\text{max\_depth}=3$ \\
                & $\text{n\_estimators}=4000$ \\
        \hline
        LGBM & $\text{learning\_rate}=0.010$ \\ 
             & $\text{max\_leaves}=63$ \\
             & $\text{n\_estimators}=5000$ \\
        \hline
        CatBoost & $\text{learning\_rate}=0.050$ \\
                 & $\text{max\_depth}=6$ \\
                 & $\text{n\_estimators}=4000$ \\
        \hline\hline\hline
    \end{tabular}
\end{table}

\begin{table}[htbp]
    \centering
    \doublerulesep=0.3pt
    \caption{パラメータ調整後のモデル評価結果}
    \label{tab:tuned_results}
    \begin{tabular}{l|p{0.8cm}p{0.8cm}p{0.8cm}|p{0.8cm}p{0.8cm}p{0.8cm}}
        \hline\hline\hline
        モデル & \multicolumn{3}{c|}{検証データ} & \multicolumn{3}{c}{テストデータ} \\
               & MSE & MAE & $R^2$ & MSE & MAE & $R^2$ \\
        \hline
        SVR       & 1096.46 & 23.31 & 0.887 & 1059.67 & 22.73 & 0.893 \\
        RFR       & 517.24  & 15.84 & 0.947 & 559.44  & 15.66 & 0.943 \\
        XGBR      & 357.44  & 13.35 & 0.963 & 529.44  & 14.51 & 0.946 \\
        LGBM      & 388.36  & 14.03 & 0.960 & 478.33  & 13.71 & 0.952 \\
        CatBoost  & 312.33  & 12.07 & 0.968 & 466.06  & 13.43 & 0.953 \\
        \hline\hline\hline
    \end{tabular}
\end{table}

表\ref{tab:default_results}の結果より,デフォルトのパラメータ設定ではテストデータに対してCatBoostが最も高い$R^2$(0.952)を示し,最も低いMSEおよびMAEを記録した.SVRは$R^2$が0.585と低く,他の手法と比較して精度が劣ることが分かった.XGBoost,LightGBM,RandomForestは$R^2$が0.93以上を達成し,いずれも高い精度を示したが,CatBoostには及ばなかった.

表\ref{tab:tuned_results}の結果より,Optunaによるハイパーパラメータの調整後,全モデルにおいて$R^2$が向上した.SVRは$R^2$が0.893まで改善し,MSEおよびMAEも大幅に低下した.XGBoostとLightGBMは検証データに対する$R^2$がそれぞれ0.963および0.960に向上し,CatBoostは$R^2$が0.968となった.テストデータに対する$R^2$は,調整後のCatBoostが0.953と最も高く,MSEおよびMAEも最も低かった.XGBoostは $R^2$が0.937から0.946へ向上し,LightGBMは0.001ポイント向上した.RandomForestは,$R^2$が0.944から0.943とわずかに低下したが,安定した性能を示した.

\subsection*{6.2 特徴量重要度の分析}
本研究では,最も高い性能を示したCatBoostについて,特徴量の重要度を分析した.特徴量の重要度は,ハイパーパラメータ調整後のCatBoostモデルを用いて算出した.その結果を図\ref{fig:feature_importance} に示す.
\begin{figure}[htbp]
    \centering
    \includegraphics[width=1.2\linewidth]{./fig/feature_importance.pdf}
    \caption{CatBoost における特徴量重要度}
    \label{fig:feature_importance}
\end{figure}

\section*{A-7. 考察}
表\ref{tab:default_results}に示したデフォルトパラメータでの結果を見ると,全体的にCatBoostが最も高い性能を示し,MSEや$R^2$の観点から他のモデルよりも優れていることが分かる.特に,SVRは$R^2$が0.585と低く,MSEやMAEも他の手法と比べて大きいため,本研究に使用するモデルとしては適していない可能性がある.次に,表\ref{tab:best_params}に示したOptunaによるハイパーパラメータ調整を行った結果,表\ref{tab:tuned_results}が示すように,すべてのモデルで性能が向上した.特に,SVRではMSEが約62\%低下し,$R^2$も0.893まで改善された.しかし,それでも他の手法と比較すると誤差が大きく,最適なモデルとは言い難い.XGBoostとLightGBMは調整後に$R^2$がそれぞれ0.946,0.952に向上し,誤差も減少した.また,RandomForestはテストデータにおいて$R^2$が0.944から0.943へわずかに低下したものの,大きな変動はなく安定した性能を示した.

以上の結果を踏まえると,本研究における最適なモデルとしては,CatBoostが有力であると考えられる.MSE,MAE,$R^2$のすべての指標で他のモデルを上回る結果を示しており,特にテストデータに対する汎化性能が優れている.ただし,計算コストやモデルの解釈性を考慮すると,RandomForestやLightGBMも実用的な選択肢となる可能性がある.

図\ref{fig:feature_importance}に示した特徴量重要度の分析結果から,
RSSIを閾値とした特徴量が重要になっていることがわかる.
これは,学生食堂が開けた空間であり,RSSIの値が安定していることが要因だと考えられる.
障害物などがあまり存在せず,利用者のみが存在する空間であるため,
結果的にRSSIの値が混雑状況のみに依存した安定した情報となり,混雑度推定に貢献したと考えられる.
\section*{A-8. まとめ}

本研究では,BLE(Bluetooth Low Energy)ビーコンを活用し,機械学習を用いた学生食堂の混雑度推定手法を提案した.本手法は,Raspberry Piによって収集されたBLEスキャンデータから得られるMACアドレス数やRSSI(受信信号強度)などの特徴量を用いて,食堂内の在席人数を回帰的に推定するものである.

評価においては,実際の食堂環境において,時間帯ごとの実測人数とBLEデータを収集し,SVR,Random Forest,XGBoost,LightGBM,CatBoostの5種類の機械学習アルゴリズムにより回帰モデルを構築した.ハイパーパラメータの調整にはOptunaを用いたベイズ最適化を適用し,汎化性能の最大化を図った.

その結果,CatBoostが最も高い決定係数$R^2$を示し,MSEおよびMAEにおいても他モデルと比較して優れた推定精度を示した.また,特徴量の重要度分析から,RSSIに閾値処理を施した特徴量が推定において特に高い寄与を示した.これは,対象とした学生食堂が壁や障害物の少ない開放的な空間であり,BLE信号の受信強度(RSSI)が安定して取得できる環境にあったことが要因と考えられる.すなわち,RSSIの値が利用者数に主に依存し,他の環境要因の影響を受けにくかったため,混雑状況を反映する指標として有効に機能したものと考えられる.

さらに,提案手法はBLEビーコンの配置やデバイスの設置が比較的容易であり,既存の通信インフラに依存せずに導入可能であるため,実環境への応用可能性が高いと考えられる.一方で,デバイス間の信号干渉や,人の移動によるRSSIの変動といった要因が推定精度に与える影響があるため,今後はより堅牢な推定モデルの構築と,長期間にわたるデータの収集・分析を通じた精度向上が課題となる.

今後の展望としては,他のセンサーデバイスとのデータ融合によるマルチモーダルな混雑推定の実現が挙げられる.

  \section*{B-1. IoTデバイスとAIを用いた課題解決}
\subsection*{1. 混雑度推定の意義}
近年,IoT(Internet of Things)技術の普及が急速に進み,
身の回りの様々なものがインターネットに接続され,情報収集と活用が可能となっている.
さらに,AI(人工知能)技術の飛躍的な進歩により,収集された大量のデータを迅速かつ高度に解析することが可能となった.
IoTデバイスは,人間が直接把握することが難しいリアルタイムの情報を客観的かつ継続的に収集することが可能であり,
AIがその情報を即座に分析し,的確な予測や判断を支援することにより,社会の多様な課題解決に有効な手段として期待されている.

特に,都市部や公共施設での「混雑度推定」は,その利便性と安全性の観点から非常に有意義である.
例えば,近年の新型コロナウイルス感染症のパンデミックにおいては,施設や公共交通機関の混雑状況をリアルタイムで把握し,
混雑を避ける行動を促すことが重要視された.
また,日常的に発生する通勤ラッシュやイベント時の混雑,災害や緊急時における安全な避難誘導においても,
精度の高い混雑状況の把握と予測は必要不可欠である.IoTデバイスを用いて混雑状況や人の流れを継続的に収集し,
AIがこれらのデータを解析して混雑の発生予測や最適な行動計画を提供することができれば,施設の運用効率が向上し,安全性や快適性が飛躍的に改善される.

本プロジェクトでは,具体的に大学の学食における混雑度推定に取り組んだ.
学食は大学生の多くが日常的に利用する施設であり,特に昼休みなど特定の時間帯に集中することによる混雑が問題となっている.
本プロジェクトでは,IoTデバイスとしてセンサーを設置し,施設内の人流データをリアルタイムで収集した.
収集したデータはAIモデルによって即座に解析され,混雑状況を数値化するとともに,ピーク時間帯を事前に予測可能とした.
これにより,利用者は混雑を避けて効率的に行動できるようになり,施設管理側も適切な人員配置やサービス運営計画を立てやすくなった.
このように,IoTとAIを組み合わせることで,社会全体の効率化,生活の質の改善を実現する可能性が広がっている.
%近年,IoT(Internet of Things)技術の普及が急速に進み,
%身の回りの様々なものがインターネットに接続され,情報収集と活用が可能となっている.
%また,AI(人工知能)技術の進展によって,これらIoTデバイスから得られる膨大なデータの解析・活用が効率化され,
%従来人間が行っていた判断や予測の精度が飛躍的に向上している.
%このようなIoTとAIの組み合わせは,医療・介護,防災・安全,都市インフラ管理,
%交通・物流など多様な社会課題の解決に寄与する可能性を持っている.
%
%特に,都市部や公共施設における「混雑度推定」は,具体的で社会的意義のある事例として注目されている.
%例えば,新型コロナウイルス感染症拡大の際には,施設内や交通機関の混雑状況をリアルタイムで把握し,人々が適切な行動を選択できる仕組みが重要視された.
%また,イベントや災害時における安全な避難誘導を実現するためにも,正確な混雑度推定技術は不可欠である.
%IoTデバイスが施設や公共空間における人の動きをデータとして収集し,それをAIが迅速かつ正確に解析することで,
%社会全体の利便性と安全性が大幅に向上することが期待されている.
\subsection*{2. Society5.0との関連}
\begin{figure}[tb]
	\centering
	\includegraphics[width=8cm]{./images/society5\_0.jpg}
	\caption{society5.0}
	\label{fig:society5.0}
\end{figure}

日本政府が提唱する「Society 5.0」\cite{society5.0}は,サイバー空間とフィジカル空間(現実世界)が高度に融合した社会を目指している.
Society 5.0では,デジタル技術を活用し,様々な社会課題の解決と経済成長を同時に実現する人間中心の社会を目標としている(図\ref{fig:society5.0}).
具体的には,膨大なデータをサイバー空間で収集・蓄積し,AI技術で解析して現実世界の課題を解決する仕組みが中心となっている.
本プロジェクトにおける学食でのIoTデバイスを活用した混雑度推定は,まさにこうしたSociety 5.0の理念を具体化した実践的な取り組みである.

Society 5.0が求めるのは,人々が安心・安全・快適に生活できる社会である.
そのためには,正確な現実世界の情報をリアルタイムで把握し,その情報に基づいてAIが迅速かつ的確な判断を下す仕組みが不可欠である.
IoTデバイスがフィジカル空間の情報をリアルタイムで収集し,サイバー空間に反映させることで,AIが得た解析結果を現実の社会に迅速にフィードバックすることが可能となる.
その結果,人々が状況に応じて適切で効率的な行動を選択できる環境が構築される.
このプロセスは,従来の経験則や人的判断に依存することなく,常に最適解を提示し続けることを目標としている.

本プロジェクトを通じ,IoTとAIの融合によって,施設管理の効率化,混雑状況の最適化を図り,
Society 5.0が掲げる高度で人間中心の未来社会の実現に貢献することを目指している.








%日本政府が提唱する「Society 5.0」は,サイバー空間とフィジカル空間(現実世界)を高度に融合させた社会を目指すものである.
%Society 5.0では,ビッグデータをAIで解析することにより,経済発展と社会的課題の解決を同時に実現する「人間中心」の社会が掲げられている.
%本プロジェクトで取り組むIoTデバイスを活用した混雑度推定は,まさにこのSociety 5.0の理念に沿ったものである.
%
%Society 5.0における重要な要素として,「データ収集」と「AIによるデータ解析」がある.
%IoTデバイスを用いたリアルタイムのデータ収集は,フィジカル空間の現状を正確にサイバー空間へと反映させる役割を担う.
%収集されたデータをAIによって解析し,混雑度などの状況を迅速に予測・判断することで,人々がリアルタイムで最適な行動を取れるよう支援する仕組みを構築できる.
%これにより,従来の経験や感覚に頼った判断から脱却し,より効率的で安全かつ持続可能な社会を実現する基盤を提供することが可能となる.
%
%さらに,このプロジェクトの取り組みは,持続可能な開発目標(SDGs)にも直接貢献することができる.
%例えば,混雑度の可視化・管理による公共空間の安全性向上(目標11:持続可能な都市),感染症の拡大防止(目標3:健康と福祉)などの目標と直接関連し,Society 5.0の推進と同時に国際社会の目標達成にも寄与することができる.
%本プロジェクトを通じ,IoTとAIの融合がもたらす新たな価値を具体的に示し,Society 5.0の理念の実現に貢献することを目指している.
\chapter*{B-2. システムの工夫}
\chapter*{B-3. 開発体制}
\subsection*{1. プロジェクト全体の体制}
本節では,プロジェクト全体の組織体制について,1.1.先行研究調査・BLE計測機器の調査および作成,1.2.混雑度推定アプリ【PALTO AI】の作成の2つに分けて述べる.

\subsubsection*{1.1. 先行研究調査・BLE計測機器の調査および作成}
BLE計測機器の作成に必要な知見を収集し、試作・検証を行う。
\begin{itemize}
    \item {\bfseries 先行研究調査}
    
    BLEを用いて混雑度推定を行った研究事例,人数推定に必要となるデータを調査する.
	さらに,BLE信号の特性や、ノイズ要因、測定誤差の補正手法についても調査を行い、計測の精度向上を図る。
    
    \item {\bfseries BLE計測機器の調査および作成}
    
    先行研究調査で調査したデータを取得するよう計測機器を設計,試作する.
    計測機器側,サーバ側に役割分担をし,以下の機能を設計・実装する。
    \begin{itemize}
        \item {\bfseries 計測機器側}: センサーデータの取得、データの前処理、送信プロトコルの実装
		測定環境の影響を考慮し、適切なセンサー配置を設計する。また、低消費電力化のための最適な通信間隔を設定する。
        \item {\bfseries サーバ側}: データの受信処理、データベースへの保存、リアルタイムデータ処理
		受信データのフィルタリングや、誤差補正アルゴリズムを実装し、より正確なデータ解析を可能にする。
    \end{itemize}
\end{itemize}

\subsubsection*{1.2. 混雑度推定アプリ【PASLTO AI】の作成}
BLE計測機器との連携を実現するアプリを開発し、システムとしての統合を図る。バックエンド,フロントエンドに役割分担し作成する.
\begin{itemize}
    \item {\bfseries フロントエンド}
    
    ユーザーインターフェースの設計と開発、BLEデータの可視化、ユーザー入力機能の実装を行う.
	ユーザーがリアルタイムで混雑度情報を取得できるよう、適切なデータ更新頻度と視覚的な表現方法を設計する。
	また、異なる端末環境(iOS/Android)に対応したレスポンシブデザインを実装する。
    
    \item {\bfseries バックエンド}
    
    APIの開発と管理、データベースの設計・最適化、BLEデータの解析処理を行う.
	データ解析アルゴリズムの開発に加え、時系列データの処理を最適化し、より正確な混雑度推定を実現する。
	さらに、負荷分散を考慮したシステム構成を設計し、多数のユーザーが同時にアクセスしても安定した動作を保証する。
\end{itemize}


\subsection*{2. プロジェクト管理}

本節では,我々がプロジェクトを円滑に進めるために使用していた管理ツールおよび進捗や問題の報告等を実現するための体制について述べる.
	
今回のプロジェクトへ取り組むにあたり,IoT 機器による BLE 取得や混雑度を可視化するシステムの処理内容を記述したソースコードの管理,
並びにタスクの管理や情報共有を実現するために以下の 3 つのツールを用いた.
	
\begin{enumerate}
	\item \textbf{GitHub}
	
	GitHub とは,Git\cite{Git} を基盤とするリポジトリ(データベース)を用いたソースコード管理と開発者同士のコラボレーションを実現するプラットフォームのことである\cite{GitHub}.
	分散型のソースコード管理では,各開発者がリモートリポジトリとは別にローカルリポジトリを個人のローカルディスクに持ち,
	ローカルリポジトリに対してコミット等の処理を行う仕組みになっているものをいう(図\ref{fig:分散型のバージョン管理}).
	なお,そのままでは開発者間でリポジトリを共通できないため,必要に応じてローカルリポジトリの内容とリモートリポジトリの内容を同期させることになる.
	\begin{figure}[tb]
		\centering
		\includegraphics[scale=0.5]{./fig/distributed\_vcs.pdf}
		\caption{分散型のバージョン管理}
		\label{fig:分散型のバージョン管理}
	\end{figure}
	
	一般に,手元のリポジトリの内容を別のリポジトリへ反映(同期)させることをプッシュ(push)といい,
	逆に別のリポジトリの内容を手元のリポジトリへ取り込む操作のことをプル(pull)という.	
	開発者がローカルリポジトリに対して行ったコミットはそのままではリモートリポジトリに対しては反映されないため,
	変更内容をリモートリポジトリへプッシュする必要がある.
	ただし,誰でも自由にリモートリポジトリへプッシュ操作を行えるわけではなく,通常はそのための権限を持った一部の開発者(管理者)に限られている.
	それゆえ多くの場合,開発者たちは変更内容を管理者へ提示し,それをリモートリポジトリへ反映してもらうよう依頼することになる.
	つまり,管理者がプル操作を行うことで依頼のあったローカルリポジトリの内容をリモートリポジトリへ取り込むことができる.
	そのような依頼のことをプルリクエスト(pull request)といい,分散型バージョン管理システムを使った開発プロジェクトではプルリクエストが 1 つの開発・保守作業単位になっている場合もある.
	
	\item \textbf{Notion}
	
	Notion とは,メモ・タスク管理・ドキュメント作成・データベース機能を統合した多機能な情報管理ツールのことである\cite{Notion}.
	具体的には,Markdown 対応のメモや To-Do リスト・カンバンボードでタスク管理ができるだけでなく,データベース機能により情報を整理・フィルタリングが可能である.
	さらに,リアルタイム共同編集にも対応しており,チームでのプロジェクト管理にも適している.
	これらのようにカスタマイズ性が高いことから,個人から企業まで幅広く活用されている.
	
	我々は,目的を達成するために必要なタスクを洗い出し,Notionの機能を活用して整理を行っていた.
	図\ref{fig:Notionを使ったタスク管理}にその一部の内容を示す.
	\begin{figure}[tb]
		\centering
		\includegraphics[scale=0.45]{./fig/notion.pdf}
		\caption{Notionを使ったタスク管理}
		\label{fig:Notionを使ったタスク管理}
	\end{figure}
	
	\item \textbf{Discord}
	
	Discord とは,ボイスチャット・テキストチャット・ビデオ通話ができるコミュニケーションツールのことである\cite{Discord}.
	サーバーを作成し,チャンネルごとにメンバーと交流できるためゲーマーやコミュニティ、ビジネス用途にも活用されている.
	
	我々は,連絡手段としてこのツールを活用した.
	例えば,進捗確認や課題の報告を行う週1回の対面定例ミーティングの告知に使用した.
	ミーティングでは、役割ごとにグループ分けされた班が、それぞれ進捗や課題を整理したプレゼン資料を作成し、報告を行った.
\end{enumerate}


\section*{3. スケジュールとマイルストーン}

  
  \small
  \bibliographystyle{jsai}
  \bibliography{ref}
\end{document}

