\section*{2. プロジェクト管理・コミュニケーション}

本節では,我々がプロジェクトを円滑に進めるために使用していた管理ツールおよび進捗や問題の報告等を実現するための体制について述べる.
\begin{itemize}
	\item \textbf{プロジェクト管理ツール}
	
	今回のプロジェクトへ取り組むにあたり,IoT 機器による BLE 取得や混雑度を可視化させるシステムの処理内容を記述したソースコードの管理,
	並びにタスクの管理を実現するために以下の 2 つのツールを用いた.
	
	\begin{enumerate}
		\item \textbf{GitHub}
		
		GitHub とは,Git\cite{Git} を基盤とするリポジトリ(データベース)を用いたソースコード管理と開発者同士のコラボレーションを実現するプラットフォームのことである\cite{GitHub}.
%		分散型のソースコード管理では,各開発者がリモートリポジトリとは別にローカルリポジトリを個人のローカルディスクに持ち,
%		ローカルリポジトリに対してコミット等の処理を行う仕組みになっているものをいう(図\ref{fig:分散型のバージョン管理}).
%		なお,そのままでは開発者間でリポジトリを共通できないため,必要に応じてローカルリポジトリの内容とリモートリポジトリの内容を同期させることになる.
%		\begin{figure}[H]
%			\centering
%			\includegraphics[scale=0.6]{./fig/distributed\_vcs.pdf}
%			\caption{分散型のバージョン管理}
%			\label{fig:分散型のバージョン管理}
%		\end{figure}
		
		\item \textbf{Notion}
		
		Notion とは,メモ・タスク管理・ドキュメント作成・データベース機能を統合した多機能な情報管理ツールのことである\cite{Notion}.
		
	\end{enumerate}
	
	\item \textbf{コミュニケーション体制}
	
	進捗確認や課題の報告等を目的として対面の定例ミーティングを週 1 日で実施した.
\end{itemize}

