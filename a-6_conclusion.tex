\section*{A-8. まとめ}

本プロジェクトでは,BLEビーコンを活用し,機械学習を用いた学生食堂の混雑度推定手法を提案した.
本手法は,Raspberry Piによって収集されたBLEスキャンデータから得られるMACアドレス数やRSSIなどの特徴量を用いて,食堂内の在席人数を回帰的に推定した.
評価においては,実際の食堂環境で時間帯ごとの実測人数とBLEデータを収集し,SVR,Random Forest,XGBoost,LightGBM,CatBoostの5種類の機械学習アルゴリズムにより回帰モデルを構築した.
ハイパーパラメータの調整にはOptunaを用いたベイズ最適化を適用し,汎化性能の最大化を図った.
その結果,CatBoostが最も高い決定係数$R^2$を示し,MSEおよびMAEにおいても他モデルと比較して優れた推定精度を示した.
また,特徴量の重要度分析から,RSSIに閾値処理を施した特徴量が推定において特に高い寄与を示した.
これは,対象とした学生食堂が壁や障害物の少ない開放的な空間であり,BLE信号のRSSIが安定して取得できる環境にあったことが要因と考えられる.
これはRSSIの値が利用者数に主に依存し,他の環境要因の影響を受けにくかったため,混雑状況を反映する指標として有効に機能したと考えられる.
さらに,本プロジェクトの提案手法はBLEビーコンの配置やデバイスの設置が比較的容易であり,既存の通信インフラに依存せずに導入可能であるため,実環境への応用可能性が高いと考えられる.
一方で,デバイス間の信号干渉や,人の移動によるRSSIの変動といった要因が推定精度に与える影響や,利用者の持つデバイスの種類や数に関して考慮ができていない.
今後はより堅牢な推定モデルの構築と,駅などのより様々な属性の人が利用する施設でのデータ収集などが必要になる.
