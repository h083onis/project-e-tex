\section*{A-8. まとめ}

本研究では,BLE(Bluetooth Low Energy)ビーコンを活用し,機械学習を用いた学生食堂の混雑度推定手法を提案した.本手法は,Raspberry Piによって収集されたBLEスキャンデータから得られるMACアドレス数やRSSI(受信信号強度)などの特徴量を用いて,食堂内の在席人数を回帰的に推定するものである.

評価においては,実際の食堂環境において,時間帯ごとの実測人数とBLEデータを収集し,SVR,Random Forest,XGBoost,LightGBM,CatBoostの5種類の機械学習アルゴリズムにより回帰モデルを構築した.ハイパーパラメータの調整にはOptunaを用いたベイズ最適化を適用し,汎化性能の最大化を図った.

その結果,CatBoostが最も高い決定係数$R^2$を示し,MSEおよびMAEにおいても他モデルと比較して優れた推定精度を示した.また,特徴量の重要度分析から,RSSIに閾値処理を施した特徴量が推定において特に高い寄与を示した.これは,対象とした学生食堂が壁や障害物の少ない開放的な空間であり,BLE信号の受信強度(RSSI)が安定して取得できる環境にあったことが要因と考えられる.すなわち,RSSIの値が利用者数に主に依存し,他の環境要因の影響を受けにくかったため,混雑状況を反映する指標として有効に機能したものと考えられる.

さらに,提案手法はBLEビーコンの配置やデバイスの設置が比較的容易であり,既存の通信インフラに依存せずに導入可能であるため,実環境への応用可能性が高いと考えられる.一方で,デバイス間の信号干渉や,人の移動によるRSSIの変動といった要因が推定精度に与える影響があるため,今後はより堅牢な推定モデルの構築と,長期間にわたるデータの収集・分析を通じた精度向上が課題となる.

今後の展望としては,他のセンサーデバイスとのデータ融合によるマルチモーダルな混雑推定の実現が挙げられる.
