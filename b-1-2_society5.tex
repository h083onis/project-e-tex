\subsection*{2. Society5.0との関連}
\begin{figure}[tb]
	\centering
	\includegraphics[width=8cm]{./images/society5\_0.jpg}
	\caption{society5.0}
	\label{fig:society5.0}
\end{figure}

日本政府が提唱する「Society 5.0」\cite{society5.0}は,サイバー空間とフィジカル空間(現実世界)が高度に融合した社会を目指している.
Society 5.0では,デジタル技術を活用し,様々な社会課題の解決と経済成長を同時に実現する人間中心の社会を目標としている(図\ref{fig:society5.0}).
具体的には,膨大なデータをサイバー空間で収集・蓄積し,AI技術で解析して現実世界の課題を解決する仕組みが中心となっている.
本プロジェクトにおける学食でのIoTデバイスを活用した混雑度推定は,まさにこうしたSociety 5.0の理念を具体化した実践的な取り組みである.

Society 5.0が求めるのは,人々が安心・安全・快適に生活できる社会である.
そのためには,正確な現実世界の情報をリアルタイムで把握し,その情報に基づいてAIが迅速かつ的確な判断を下す仕組みが不可欠である.
IoTデバイスがフィジカル空間の情報をリアルタイムで収集し,サイバー空間に反映させることで,AIが得た解析結果を現実の社会に迅速にフィードバックすることが可能となる.
その結果,人々が状況に応じて適切で効率的な行動を選択できる環境が構築される.
このプロセスは,従来の経験則や人的判断に依存することなく,常に最適解を提示し続けることを目標としている.

本プロジェクトを通じ,IoTとAIの融合によって,施設管理の効率化,混雑状況の最適化を図り,
Society 5.0が掲げる高度で人間中心の未来社会の実現に貢献することを目指している.








%日本政府が提唱する「Society 5.0」は,サイバー空間とフィジカル空間(現実世界)を高度に融合させた社会を目指すものである.
%Society 5.0では,ビッグデータをAIで解析することにより,経済発展と社会的課題の解決を同時に実現する「人間中心」の社会が掲げられている.
%本プロジェクトで取り組むIoTデバイスを活用した混雑度推定は,まさにこのSociety 5.0の理念に沿ったものである.
%
%Society 5.0における重要な要素として,「データ収集」と「AIによるデータ解析」がある.
%IoTデバイスを用いたリアルタイムのデータ収集は,フィジカル空間の現状を正確にサイバー空間へと反映させる役割を担う.
%収集されたデータをAIによって解析し,混雑度などの状況を迅速に予測・判断することで,人々がリアルタイムで最適な行動を取れるよう支援する仕組みを構築できる.
%これにより,従来の経験や感覚に頼った判断から脱却し,より効率的で安全かつ持続可能な社会を実現する基盤を提供することが可能となる.
%
%さらに,このプロジェクトの取り組みは,持続可能な開発目標(SDGs)にも直接貢献することができる.
%例えば,混雑度の可視化・管理による公共空間の安全性向上(目標11:持続可能な都市),感染症の拡大防止(目標3:健康と福祉)などの目標と直接関連し,Society 5.0の推進と同時に国際社会の目標達成にも寄与することができる.
%本プロジェクトを通じ,IoTとAIの融合がもたらす新たな価値を具体的に示し,Society 5.0の理念の実現に貢献することを目指している.