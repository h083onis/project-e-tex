\section*{3. バックエンドの工夫点}

続いて,バックエンドの工夫点について述べる.

\subsection*{(1)Flaskの採用}
バックエンド開発には,Python の Web フレームワークである Flask を採用した.Flask は、以下の点で本システムに適していると判断した。

\begin{itemize}
	\item シンプルで使いやすい
	
	シンプルで柔軟な設計により,少ないコードで API を構築できる.そのため,学習コストが低く,迅速なプロトタイプ作成や機能追加が可能となる.
	
	\item 軽量
	
	必要最低限の機能のみ提供するため,動作が軽い.そのため,小規模なアプリケーション開発に適しており,リソースの限られた環境でも利用しやすい.

	\item フロントエンドとの連携が容易
	 
	フロントエンドで必要な API エンドポイントを迅速に構築でき,クライアントからのリクエストに柔軟に対応できる.また,JSON 形式のデータを簡単に送受信できるため,フロントエンドと異なるフレームワークを使用しても連携をスムーズに行える.
\end{itemize}

\subsection*{(2)データベースの採用}
本システムでは,モデルによる予測結果の保存方法として,データベース設計を採用した.これにより,拡張性や永続性を確保し,今後の機能追加やデータ増加に柔軟に対応できる設計を実現した.表\ref{tbl:prediction_table}に示すように、予測結果は適切なスキーマ設計を通じて格納され,データの整合性を保ちながら,将来的な拡張にも対応可能な構成としている.

\begin{table}[tb]
	\centering
	\caption{予測結果を保存するテーブル}
	\label{tbl:prediction_table}
	\small
	\doublerulesep=0.3pt
	\begin{tabular}{l|l|l|l} \hline\hline\hline
		カラム名 & データ型 & 制約 & 説明 \\ \hline
		id & INT & PRIMARY KEY, &  識別子 \\
		&&AUTO& \\
		&&INCREMENT& \\ \hline
		timestamp & DATETIME & DEFAULT  & スキャン時刻 \\
		&&CURRENT\_TIME& \\
		&&STAMP& \\ \hline
		other\_data & JSON & NOT NULL & 位置情報,RSSIなど \\  \hline\hline\hline
	\end{tabular}
\end{table}

\subsection*{(3)機能のAPI化}
本システムでは,バックエンド側で提供する機能を API として切り出し,フロントエンドとバックエンドの分離を図った.
このアプローチにより,次のようなメリットが得られた.

\begin{itemize}
	\item フロントエンドとバックエンドの独立性の向上
	
	フロントエンドとバックエンドがAPIを通じて通信する構造にすることで,フロントエンド側の実装に依存せずにバックエンドの開発が可能になった.
	
	\item 再利用性の向上
	 
	各機能がAPIとして分離されることで,その機能を他のシステムやサービスで再利用することが容易になった.
	
\end{itemize}

本システムで作成したAPIの概要について,表\ref{tbl:API}に示す.

\begin{table}[tb]
	\centering
	\caption{作成したAPI}
	\label{tbl:API}
	\small
	\doublerulesep=0.3pt
    \begin{tabular}{l|l|p{6cm}} \hline\hline\hline
		API名 & 概要 & レスポンスの例 \\ \hline
		get\_prediction & 最新の予測結果を取得 & {"prediction": 2.0, "timestamp": "Tue, 14 Jan 2025 23:11:50 GMT"} \\ \hline
		insert\_scanned\_data & スキャンデータをデータベースに保存 & {"message": "Data inserted successfully"} \\ \hline\hline\hline
	\end{tabular}
\end{table}