\subsection*{1. プロジェクト全体の体制}
本節では,プロジェクト全体の組織体制について,1.1.先行研究調査・BLE計測機器の調査および作成,1.2.混雑度推定アプリ【PALTO AI】の作成の2つに分けて述べる.

\subsubsection*{1.1. 先行研究調査・BLE計測機器の調査および作成}
BLE計測機器の作成に必要な知見を収集し、試作・検証を行う。
\begin{itemize}
    \item {\bfseries 先行研究調査}
    
    BLEを用いて混雑度推定を行った研究事例,人数推定に必要となるデータを調査する.
	さらに,BLE信号の特性や、ノイズ要因、測定誤差の補正手法についても調査を行い、計測の精度向上を図る。
    
    \item {\bfseries BLE計測機器の調査および作成}
    
    先行研究調査で調査したデータを取得するよう計測機器を設計,試作する.
    計測機器側,サーバ側に役割分担をし,以下の機能を設計・実装する。
    \begin{itemize}
        \item {\bfseries 計測機器側}: センサーデータの取得、データの前処理、送信プロトコルの実装
		測定環境の影響を考慮し、適切なセンサー配置を設計する。また、低消費電力化のための最適な通信間隔を設定する。
        \item {\bfseries サーバ側}: データの受信処理、データベースへの保存、リアルタイムデータ処理
		受信データのフィルタリングや、誤差補正アルゴリズムを実装し、より正確なデータ解析を可能にする。
    \end{itemize}
\end{itemize}

\subsubsection*{1.2. 混雑度推定アプリ【PALTO AI】の作成}
BLE計測機器との連携を実現するアプリを開発し、システムとしての統合を図る。バックエンド,フロントエンドに役割分担し作成する.
\begin{itemize}
    \item {\bfseries フロントエンド}
    
    ユーザーインターフェースの設計と開発、BLEデータの可視化、ユーザー入力機能の実装を行う.
	ユーザーがリアルタイムで混雑度情報を取得できるよう、適切なデータ更新頻度と視覚的な表現方法を設計する。
	また、異なる端末環境(iOS/Android)に対応したレスポンシブデザインを実装する。
    
    \item {\bfseries バックエンド}
    
    APIの開発と管理、データベースの設計・最適化、BLEデータの解析処理を行う.
	データ解析アルゴリズムの開発に加え、時系列データの処理を最適化し、より正確な混雑度推定を実現する。
	さらに、負荷分散を考慮したシステム構成を設計し、多数のユーザーが同時にアクセスしても安定した動作を保証する。
\end{itemize}

