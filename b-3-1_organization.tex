\subsection*{1. プロジェクト全体の体制}
本節では,プロジェクト全体の組織体制について,1.1.先行研究調査・BLE計測機器の調査および作成,1.2.混雑度推定アプリ【PALTO AI】の作成の2つに分けて述べる.

\subsubsection*{1.1. 先行研究調査・BLE計測機器の調査および作成}
BLE計測機器の作成に必要な知見を収集し,試作・検証を行う.

\paragraph*{1.1.1 先行研究調査}\mbox{}\\
\indent BLEを用いて混雑度推定を行った研究事例,人数推定に必要となるデータを調査する.
	さらに,BLE信号の特性や,ノイズ要因,測定誤差の補正手法についても調査を行い,計測の精度向上を図る.
    
\paragraph*{1.1.2 BLE計測機器の調査および作成}\mbox{}\\
\indent 先行研究調査で調査したデータを取得するよう計測機器を設計,試作する.計測機器側,サーバ側に役割分担をし,以下の機能を設計・実装する.
   
\begin{enumerate}
    \item {\bfseries 計測機器側}: センサーデータの取得,データの前処理,送信プロトコルの実装
    測定環境の影響を考慮し,適切なセンサー配置を設計する.また,低消費電力化のための最適な通信間隔を設定する.
    \item {\bfseries サーバ側}: データの受信処理,データベースへの保存,リアルタイムデータ処理
    受信データのフィルタリングや,誤差補正アルゴリズムを実装し,より正確なデータ解析を可能にする.
\end{enumerate}

\subsubsection*{1.2. 混雑度推定アプリ【PALTO AI】の作成}
BLE計測機器との連携を実現するアプリを開発し,システムとしての統合を図る.バックエンド,フロントエンドに役割分担し作成する.
\begin{enumerate}
    \item {\bfseries フロントエンド}
    
    ユーザーインターフェースの設計と開発,BLEデータの可視化,ユーザー入力機能の実装を行う.
	ユーザーがリアルタイムで混雑度情報を取得できるよう,適切なデータ更新頻度と視覚的な表現方法を設計する.
	また,異なる端末環境(iOS/Android)に対応したレスポンシブデザインを実装する.
    
    \item {\bfseries バックエンド}
    
    APIの開発と管理,データベースの設計・最適化,BLEデータの解析処理を行う.
	データ解析アルゴリズムの開発に加え,時系列データの処理を最適化し,より正確な混雑度推定を実現する.
	さらに,負荷分散を考慮したシステム構成を設計し,多数のユーザーが同時にアクセスしても安定した動作を保証する.
\end{enumerate}

