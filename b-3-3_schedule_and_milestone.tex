\subsection*{3. スケジュールとマイルストーン}

本節では,本プロジェクトの進行スケジュールおよび各段階のマイルストーンについて表\ref{tbl:スケジュールとマイルストーン}を用いて説明する.

\begin{table}[tp]
	\centering
	\caption{スケジュールとマイルストーン}
	\label{tbl:スケジュールとマイルストーン}
	\begin{tabular}{cc} \hline
		期間 & 取り組み \\ \hline
		2024/06 & プロジェクト開始\\
		2024/07 $\sim$ 08 & IoT 機器で BLE を取得する試み\\
		2024/09 & 学食で BLE の取得実験\\
		2024/10 $\sim$ 11 & 機械学習モデルの構築と性能評価\\
		2024/12 $\sim$ 2025/01 & プロトタイプシステムの構築\\ 
		2025/02 & 学食でシステムの試運転\\
		2025/03 & 報告資料の作成\\ \hline
	\end{tabular}
\end{table}

表\ref{tbl:スケジュールとマイルストーン}に示す通り,本プロジェクトは BLE データの取得実験,機械学習モデルの構築,プロトタイプの開発など段階的に進行していった.
2025年2月には,実際の運用環境である学食においてプロトタイプシステムの試運転を実施し,その有効性を検証した.
最終的に,これまでの取り組みをまとめた報告資料の作成し,プロジェクトを完了した.




